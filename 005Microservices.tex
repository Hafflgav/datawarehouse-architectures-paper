\section{Introduction to Micro-service Architectures}
\label{sec:techKnowHow}

\subsection{Distinction between SOA and Microservice-Architectures}
Microservice Architecture is a quite commonly used buzz-word for a state of the art way of how to implement software. Having a look into the data of Google Trends it is possible to see that the popularity of this approach increases dramatically since 2014. \cite{microservices}\newline
This section aims to explain this buzz-word in the context of self-contained systems. Afterwards we will have a look upon an event based communication between such services.\newline
\\
Having monolithic application which grew over time often lead into a high maintainability and complexity. To avoid this the architecture needs to be dezerialized into multiple services, which results into a Service-Oriented Architecture (SOA). But due to current development it seems that SOA will be replaced by micorservice architectures.\cite{mircorVSsoa}\newline
Both of those methodologies suggest the decomposition of software. Therefore SOA focuses upon integrating those services for the entire company with a centralised management and governance. Instead, microservices focus on decomposing services without such global governance. This results into higher autonomy and decoupling.\cite{mircorVSsoa}\newline


\subsection{The Sense behind Self-Contained-Systems}


\subsection{Integrating an Event-Based Bus into the Microservice Landscape}
