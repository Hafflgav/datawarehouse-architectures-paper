\section{Introduction}
''The world’s most valuable resource is no longer oil, but data'' \cite{dataIsOil}\newline
This was stated in an article published by 'The Economist' in 2017 and expresses quite well how important and valuable data currently is. In the last past three years this can definitively be validated. This can lead to negative incidents as seen 2016 during the Brexit referendum or regarding the Cambridge Analytica scandal. But next to that a lot of companies try to generate customer benefits using this data.\newline
To achieve this, there is a need for special systems handling such an intensive amount of data. Basic requirements are gathering of data from heterogeneous sources, transformation and enrichment as well as the deployment for analytical applications. Of course are non-functional requirements as performance or real-time actuation also necessary while talking about specifications.\newline
\\
After pointing out the intend, why those systems are needed and introducing the requirements, lets focus on the content of this paper.\newline 
As said in the abstract it will focus on architectural fundamentals which are used for this systems. Therefore the reference architecture of Data Warehouse Systems is introduced first of all. Afterwards we will look upon some other commonly used architectures which use some newer ideas.\newline
Therefore micro-services and some techniques e.g. self-contained-systems are introduced. Later on the idea of event streams in system landscapes is explained.\newline
Having elaborated those ideas, they will be adapted and combined into possible architectures. In this section there will be an in-depth discussion about the advantages and disadvantages of each technology.\newline
Now lets have a look upon the idea of Business Process Management. The underlying process-driven-architecture enables us to erase some of those previously shown disadvantages. In the end the resulting architecture will be, as described in the title, a process-driven semi-event-based micro-service system, enabling us to run data intensive applications. 