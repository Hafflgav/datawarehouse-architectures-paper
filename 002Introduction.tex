\section{Introduction}
\label{sec:intro}
\begin{quote}
    ''The world’s most valuable resource is no longer oil, but data'' \cite{dataIsOil}
\end{quote}
This was stated in an article published by ''The Economist'' in 2017 and expresses quite well how important and valuable data currently is. In the past three years this thesis can definitively be validated. This can lead to negative incidents as seen in 2016 during the Brexit referendum or with the Cambridge Analytica scandal. A lot of companies try to generate customer benefits using historicised data.\newline
To achieve this, there is a need for special systems handling an intensive amount of data. Basic requirements are the gathering of data from heterogeneous sources, the transformation and enrichment as well as the deployment for analytical applications. Of course are non-functional requirements as performance or real-time actuation also necessary while talking about specifications.\newline
\\
After pointing out the intent, the need and requirements for those systems, let us focus on the content. In this paper, the focus is set on architectural fundamentals which can be used for those systems. First the reference architecture of Data Warehouse Systems is introduced. Afterwards we will look onto other commonly used architectures which make use of some newer ideas.
The main objective during this research is to identify if commonly used legacy architectures can be replaced by newer technologies which comes along with some advantages.
\newline
Therefore, microservices and some techniques such as self contained systems are introduced. Later on, the idea of event streams in system landscapes is explained.\newline
Having elaborated those ideas, they will be adapted and combined into possible architectures. In this section there will be an in-depth discussion about the advantages and disadvantages of each technology.\newline
Afterwards, there will be an outlook on the idea of Business Process Management. The underlying process driven architecture enables us to erase some previously shown disadvantages.\newline
\\
Furthermore, this will result into the hypothesis of this paper. Throughout the adaption of process-driven microservice principles in the area of Data Warehouse architectures, a huge improvement concerning individual characteristics is possible even with relatively little effort. This advance will lead to an optimised way of presenting various results to the user without increasing the systems complexity.\newline 
In the end, the resulting architecture will be (as described within the title) a process-driven semi-event-based microservice system, enabling us to run data-intensive applications. \newline
As we make progress in this field of research, we will make use of several other publications and expert interviews to underpin this paper. 